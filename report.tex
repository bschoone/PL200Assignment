\documentclass[a4paper,12pt,notitlepage]{article}
\usepackage{graphicx}
\graphicspath{ {images/} }
\usepackage[utf8]{inputenc}
\usepackage{parskip}
\usepackage[utf8]{inputenc}
\usepackage{listings}
\usepackage{color}

\title{Programming Languages 200 Assignment Report}
\author{Bradley Jay Schoone\\Student ID: 16144198}

\date{October, 2014}

\begin{document}
\maketitle

\begin{center}
Practical Time: Wednesday, 4pm \- 5pm
\end{center}

\begin{center}
I, Bradley Jay Schoone (16144198), certify that this document and all code relating to it was written by myself, unless otherwise stated.
\end{center}
\clearpage
\section{Usage}
\begin{itemize}
  \item Firstly, run ``make clean" to remove the old compiled files files and log files.
  \item Then simply run ``make" to compile the program.
\end{itemize}

\section{Structure of My Source Code}

\subsection{EBNF}
Located in the \texttt{EBNF.txt} file.
The EBNF follows the ISO BNF Standard.

\subsection{Lex}
Located in the \texttt{assignment.l} file.

The single character tokens simply return themselves for simplicity's sake, except for the \texttt{;} character, because after 4 seconds of trying to escape the character so it could be used as a token , I couldn't figure out how, so I simply called it \texttt{\_SEMICOLON\_}.

This was pretty straight forward.


\subsection{Yacc}
Located in the \texttt{assignment.y} file.

Yacc is not very much fun at all.

This Yacc file differs from my EBNF slightly, as I needed a few more rules in certain cases to make sure that it matched my EBNF correctly and functioned as described in the Syntax Diagram (See (3) Design Issues).


\section{Design Issues}



\subsection{Infinite Loops in the EBNF}
Some of the cases specified in the Syntax Diagram allow for a seemingly infinite loops.
For example with the \texttt{variable\_declaration}, the language is allowed to have as many "ident : ident" separated by commas as it pleases.

This made translating to Yacc difficult. Thus, using Left-Recursion, i needed to add another rule to make sure the language worked correctly.

\subsection{Syntactic Sugar in the Language}
Another thing I had to cater for was syntactic consistency.
I allowed for identifiers to be only lowercase and for reserved words, such as END, WHILE, BEGIN etc to be uppercase. 

Reason being, there might be some confusion syntactically as to what was a reserved word and what wasn't if mixed casing was allowed. Plus, it also make the languages source code much more readable.

In my Lex, i also ignore whitespace. So the output lines up nicely and doesnt look all too awful.

\section{References}
\begin{itemize}
    \item lex \& yacc, 2nd Edition by O'Reilly Media: 
    
    \texttt{http://shop.oreilly.com/product/9781565920002.do}
    
    \item Lex and YACC Primer by Bert Hubert:
    
    \texttt{http://ds9a.nl/lex-yacc/cvs/lex-yacc-howto.html}
    \item Curtin University Programming Languages 200 Lecture Materials 
    
\end{itemize}

\end{document}
